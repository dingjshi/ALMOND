\documentclass[a4paper]{book}
\usepackage[times,inconsolata,hyper]{Rd}
\usepackage{makeidx}
\usepackage[utf8]{inputenc} % @SET ENCODING@
% \usepackage{graphicx} % @USE GRAPHICX@
\makeindex{}
\begin{document}
\chapter*{}
\begin{center}
{\textbf{\huge ALMOND}}
\par\bigskip{\large \today}
\end{center}
\begin{description}
\raggedright{}
\inputencoding{utf8}
\item[Type]\AsIs{Package}
\item[Title]\AsIs{Bayesian Analysis of Late (Local Average Treatment Effect) for Missing Or/and Nonnormal Data (ALMOND)}
\item[Version]\AsIs{0.1.0}
\item[Author]\AsIs{Dingjing Shi [aut, cre],
Xin Tong [ctb],
M. Joseph Meyer [ctb]}
\item[Maintainer]\AsIs{Dingjing Shi }\email{ds4ue@virginia.edu}\AsIs{}
\item[Description]\AsIs{Using Bayesian robust two-stage causal models with instrumental variables to estimate the Local Average Treatment Effect and simultaneously handle the nonnormal and missing data.}
\item[License]\AsIs{GPL (>= 3.0)}
\item[Encoding]\AsIs{UTF-8}
\item[LazyData]\AsIs{true}
\item[Imports]\AsIs{R2OpenBUGS, coda, Formula}
\item[Suggests]\AsIs{knitr, rmarkdown}
\item[RoxygenNote]\AsIs{6.1.1}
\end{description}
\Rdcontents{\R{} topics documented:}
\inputencoding{utf8}
\HeaderA{gts.nnormal}{Apply the generalized Bayesian two-stage normal-based causal model with instrumental variables.}{gts.nnormal}
%
\begin{Description}\relax
The \code{gts.nnormal} function applies the generalized Bayesian two-stage
normal-based causal model to the categorical treatment data.
The model best suits the normally-distributed outcome data
that are complete or ignorably missing (i.e., missing completely at random or missing at random).
\end{Description}
%
\begin{Usage}
\begin{verbatim}
gts.nnormal(formula, data, advanced = FALSE, adv.model, b0 = 1,
  B0 = 1e-06, g0 = 0, G0 = 1e-06, e0 = 0.001, E0 = 0.001,
  beta.start = NULL, gamma.start = NULL, e.start = NULL,
  n.chains = 1, n.burnin = floor(n.iter/2), n.iter = 10000,
  n.thin = 1, DIC, debug = FALSE, codaPkg = FALSE)
\end{verbatim}
\end{Usage}
%
\begin{Arguments}
\begin{ldescription}
\item[\code{formula}] An object of class formula: a symbolic description of the model to be fitted.
The details of the model specification are given under "Details".

\item[\code{data}] A dataframe with the variables to be used in the model.

\item[\code{advanced}] Logical; if FALSE (default), the model is specified using the formula argument,
if TRUE, self-defined models can be specified using the adv.model argument.

\item[\code{adv.model}] Specify the self-defined model. Used when advanced=TRUE.

\item[\code{b0}] The mean hyperparameter of the normal distribution (prior distribution)
for the first-stage generalized causal model coefficients, i.e., coefficients for the instrumental variables.
This can either be a numerical value or a vector with dimensions equal to the number of coefficients
for the instrumental variables. If this takes a numerical value, then that values will
serve as the mean hyperparameter for all of the coefficients for the instrumental variables.
Default value of 0 is equivalent to a noninformative prior for the normal distributions.
Used when advanced=FALSE.

\item[\code{B0}] The precision hyperparameter of the normal distribution (prior distribution)
for the first stage generalized causal model coefficients.
This can either be a numerical value or a vector with dimensions equal to the number of coefficients
for the instrumental variables. If this takes a numerical value, then that values will
serve as the precision hyperparameter for all of the coefficients for the instrumental variables.
Default value of 1.0E-6 is equivalent to a noninformative prior for the normal distributions.
Used when advanced=FALSE.

\item[\code{g0}] The mean hyperparameter of the normal distribution (prior distribution)
for the second-stage generalized causal model coefficients,
i.e., coefficients for the treatment variable and other regression covariates).
This can either be a numerical value if there is only one treatment variable in the model,
or a if there is a treatment variable and multiple regression covariates,
with dimensions equal to the total number of coefficients for the treatment variable and covariates.
Default value of 0 is equivalent to a noninformative prior for the normal distributions.
Used when advanced=FALSE.

\item[\code{G0}] The precision hyperparameter of the normal distribution (prior distribution)
for the second-stage generalized causal model coefficients.
This can either be a numerical value if there is only one treatment variable in the model,
or a vector if there is a treatment variable and multiple regression covariates,
with dimensions equal to the total number of coefficients for the treatment variable and covariates.
Default value of 1.0E-6 is equivalent to a noninformative prior for the normal distributions.
Used when advanced=FALSE.

\item[\code{e0}] The location hyperparameter of the inverse Gamma distribution (prior for the variance of the
normal distribution on the model residual).
Default of 0.001 is equivalent to the noninformative prior for the inverse Gamma distribution.

\item[\code{E0}] The shape hyperparameter of the inverse Gamma distribution (prior for the variance of the
normal distribution on the model residual).
Default of 0.001 is equivalent to the noninformative prior for the inverse Gamma distribution.

\item[\code{beta.start}] The starting values for the first-stage generalized causal model coefficients,
i.e., coefficients for the instrumental variables.
This can either be a numerical value or a column vector with dimensions
equal to the number of first-stage coefficients.
The default value of NA will use the IWLS (iteratively reweighted least squares) estimate
of first-stage coefficients as the starting value.
If this is a numerical value, that value will
serve as the starting value mean for all the first-stage beta coefficients.

\item[\code{gamma.start}] The starting values for the second-stage generalized causal model coefficients,
i.e., coefficients for the treatment variable and the model covariates.
This can either be a numerical value or a column vector with dimensions
equal to the number of second-stage coefficients.
The default value of NA will use the IWLS (iteratively reweighted least squares) estimate
of second-stage coefficients as the starting value.
If this is a numerical value, that value will
serve as the starting value mean for all the second-stage gamma coefficients.

\item[\code{e.start}] The starting value for the precision hyperparameter of the inverse gamma distribution
(prior for the scale parameter of Student's t distribution of the model residual).
The default value of NA will use the inverse of the residual variance from the
IWLS (iteratively reweighted least square) estimate of the second-stage model.

\item[\code{n.chains}] The number of Markov chains. The default is 1.

\item[\code{n.burnin}] Length of burn in, i.e., number of iterations to discard at the beginning.
Default is n.iter/2, that is, discarding the first half of the simulations.

\item[\code{n.iter}] The number of total iterations per chain (including burnin). The default is 10000.

\item[\code{n.thin}] The thinning rate. Must be a positive integer. The default is 1.

\item[\code{DIC}] Logical; if TRUE (default), compute deviance, pD, and DIC. The rule pD=Dbar-Dhat is used.

\item[\code{codaPkg}] Logical; if FALSE (default), an object is returned; if TRUE,
file names of the output are returned.
\end{ldescription}
\end{Arguments}
%
\begin{Details}\relax
\begin{enumerate}

\item Bayesian two-stage causal models are specified symbolically.
A typical model has the form \emph{reponse \textasciitilde{} terms|instrumental\_variables},
where response is the (numeric) response vector and terms is a series of terms
which specifies a linear predictor (i.e., the treatment variable and the covariates) for the response.
The first specification in the term is always the treatment variable and
the remaining specifications are always the covariates for the response.
\item DIC is computed as \emph{mean(deviance)+pD}.
\item Prior distributions used in ALMOND.
\begin{itemize}

\item Generalized causal model coefficients at both stages: normal distributions.
\item The generalized causal model residual term: normal distribution.

\end{itemize}


\end{enumerate}

\end{Details}
%
\begin{Value}
If \emph{codaPkg=FALSE}(default), returns an object containing summary statistics of
the saved parameters, including
\begin{ldescription}
\item[\code{s1.intercept}] Estimate of the intercept from the first stage.
\item[\code{s1.slopeP}] Estimate of the pth slope from the first stage. 
\item[\code{s2.intercept}] Estimate of the intercept from the second stage.
\item[\code{s2.slopeP}] Estimate of the pth slope from the second stage (the first slope is always
the \strong{LATE}).
\item[\code{var.e.s2}] Estimate of the residual variance at the second stage.
\item[\code{DIC}] Deviance Information Criterion.
\end{ldescription}
If \emph{codaPkg=TRUE}, the returned value is the path for the output file
containing the Markov chain Monte Carlo output.
\end{Value}
%
\begin{References}\relax
Gelman, A., Carlin, J.B., Stern, H.S., Rubin, D.B. (2003).
\emph{Bayesian data analysis}, 2nd edition. Chapman and Hall/CRC Press.

Spiegelhalter, D. J., Thomas, A., Best, N. G., Gilks, W., \& Lunn, D. (1996).
BUGS: Bayesian inference using Gibbs sampling.
\Rhref{http://www.mrc-bsu.cam.ac.uk/bugs}{}
\end{References}
%
\begin{Examples}
\begin{ExampleCode}

# Run the model
model1 <- gts.nnormal(outcome~treatment|instrument,data=simCatNormMCAR)

# Run the model with the self-defined advanced feature

model2 <- gts.nnormal(outcome~treatment|instrument,data=simCatNormMCAR,
advanced=TRUE, adv.model=my.model)

# Extract the model DIC
model1$DIC

# Extract the MCMC output
model3 <- gts.nnormal(outcome~treatment|instrument,data=simCatNormMCAR,codaPkg=TRUE)


\end{ExampleCode}
\end{Examples}
\inputencoding{utf8}
\HeaderA{gts.nnormal.s}{Apply the generalized Bayesian two-stage normal-selection causal model with instrumental variables.}{gts.nnormal.s}
%
\begin{Description}\relax
The \code{gts.nnormal.s} function applies the generalized Bayesian two-stage
normal-selection causal model to the categorical treatment data.
The model best suits the normally-distributed outcome data that are normally-distributed
and nonignorably missing (i.e., MNAR) (e.g., dropout, attrition).
\end{Description}
%
\begin{Usage}
\begin{verbatim}
gts.nnormal.s(formula, data, m.ind, advanced = FALSE, adv.model,
  b0 = 0, B0 = 1e-06, g0 = 0, G0 = 1e-06, e0 = 0.001,
  E0 = 0.001, beta.start = NULL, gamma.start = NULL,
  e.start = NULL, lambda0.start = 1, lambda1.start = 1,
  n.chains = 1, n.burnin = floor(n.iter/2), n.iter = 50000,
  n.thin = 1, DIC, debug = FALSE, codaPkg = FALSE)
\end{verbatim}
\end{Usage}
%
\begin{Arguments}
\begin{ldescription}
\item[\code{formula}] An object of class formula: a symbolic description of the model to be fitted.
The details of the model specification are given under "Details".

\item[\code{data}] A dataframe with the variables to be used in the model.

\item[\code{advanced}] Logical; if FALSE (default), the model is specified using the formula argument,
if TRUE, self-defined models can be specified using the adv.model argument.

\item[\code{adv.model}] Specify the self-defined model. Used when advanced=TRUE.

\item[\code{b0}] The mean hyperparameter of the normal distribution (prior distribution)
for the first-stage generalized causal model coefficients, i.e., coefficients for the instrumental variables.
This can either be a numerical value or a vector with dimensions equal to the number of coefficients
for the instrumental variables. If this takes a numerical value, then that values will
serve as the mean hyperparameter for all of the coefficients for the instrumental variables.
Default value of 0 is equivalent to a noninformative prior for the normal distributions.
Used when advanced=FALSE.

\item[\code{B0}] The precision hyperparameter of the normal distribution (prior distribution)
for the first stage generalized causal model coefficients.
This can either be a numerical value or a vector with dimensions equal to the number of coefficients
for the instrumental variables. If this takes a numerical value, then that values will
serve as the precision hyperparameter for all of the coefficients for the instrumental variables.
Default value of 1.0E-6 is equivalent to a noninformative prior for the normal distributions.
Used when advanced=FALSE.

\item[\code{g0}] The mean hyperparameter of the normal distribution (prior distribution)
for the second-stage generalized causal model coefficients,
i.e., coefficients for the treatment variable and other regression covariates.
This can either be a numerical value if there is only one treatment variable in the model,
or a if there is a treatment variable and multiple regression covariates,
with dimensions equal to the total number of coefficients for the treatment variable and covariates.
Default value of 0 is equivalent to a noninformative prior for the normal distributions.
Used when advanced=FALSE.

\item[\code{G0}] The precision hyperparameter of the normal distribution (prior distribution)
for the second-stage generalized causal model coefficients.
This can either be a numerical value if there is only one treatment variable in the model,
or a vector if there is a treatment variable and multiple regression covariates,
with dimensions equal to the total number of coefficients for the treatment variable and covariates.
Default value of 1.0E-6 is equivalent to a noninformative prior for the normal distributions.
Used when advanced=FALSE.

\item[\code{e0}] The location hyperparameter of the inverse Gamma distribution (prior for the variance of the
normal distribution on the model residual).
Default of 0.001 is equivalent to the noninformative prior for the inverse Gamma distribution.

\item[\code{E0}] The shape hyperparameter of the inverse Gamma distribution (prior for the variance of the
normal distribution on the model residual).
Default of 0.001 is equivalent to the noninformative prior for the inverse Gamma distribution.

\item[\code{beta.start}] The starting values for the first-stage generalized causal model coefficients,
i.e., coefficients for the instrumental variables.
This can either be a numerical value or a column vector with dimensions
equal to the number of first-stage coefficients.
The default value of NA will use the IWLS (iteratively reweighted least squares) estimate
of first-stage coefficients as the starting value.
If this is a numerical value, that value will
serve as the starting value mean for all the first-stage beta coefficients.

\item[\code{gamma.start}] The starting values for the second-stage generalized causal model coefficients,
i.e., coefficients for the treatment variable and the model covariates.
This can either be a numerical value or a column vector with dimensions
equal to the number of second-stage coefficients.
The default value of NA will use the IWLS (iteratively reweighted least squares) estimate
of second-stage coefficients as the starting value.
If this is a numerical value, that value will
serve as the starting value mean for all the second-stage gamma coefficients.

\item[\code{e.start}] The starting value for the precision hyperparameter of the inverse gamma distribution
(prior for the variance of the normal distribution on the model residual).
The default value of NA will use the inverse of the residual variance from the
IWLS (iteratively reweighted least square) estimate of the second-stage model.

\item[\code{lambda0.start}] The starting value for the intercept of the coefficient of the added-on selection model.

\item[\code{lambda1.start}] The starting value for the slope of the coefficient of the added-on selection model.

\item[\code{n.chains}] The number of Markov chains. The default is 1.

\item[\code{n.burnin}] Length of burn in, i.e., number of iterations to discard at the beginning.
Default is n.iter/2, that is, discarding the first half of the simulations.

\item[\code{n.iter}] The number of total iterations per chain (including burnin). The default is 50000.

\item[\code{n.thin}] The thinning rate. Must be a positive integer. The default is 1.

\item[\code{DIC}] Logical; if TRUE (default), compute deviance, pD, and DIC. The rule pD=Dbar-Dhat is used.

\item[\code{codaPkg}] Logical; if FALSE (default), an object is returned; if TRUE,
file names of the output are returned.

\item[\code{l0}] The mean hyperparameter of the normal distribution (prior for the added-on selection model coefficients).
This can either be a numerical value or a vector with dimensions equal to the number of coefficients for the instrumental variables.
If this takes a numerical value, then that values will serve as the mean hyperparameter for all of the coefficients
for the instrumental variables. Default value of 0 is equivalent to a noninformative prior for the normal distributions.
Used when advanced=FALSE.

\item[\code{L0}] The precision hyperparameter of the normal distribution (prior for the added-on selection model coefficients).
This can either be a numerical value or a vector with dimensions equal to the number of coefficients for the instrumental variables.
If this takes a numerical value, then that values will serve as the precision hyperparameter for all of the coefficients
for the instrumental variables.
Default value of 1.0E-6 is equivalent to a noninformative prior for the normal distributions.
Used when advanced=FALSE.
\end{ldescription}
\end{Arguments}
%
\begin{Details}\relax
\begin{enumerate}

\item The formula takes the form \emph{response \textasciitilde{} terms|instrumental\_variables}.
\code{\LinkA{gts.nnormal}{gts.nnormal}} provides a detailed description of the formula rule.
\item DIC is computed as \emph{mean(deviance)+pD}.
\item Prior distributions used in ALMOND.
\begin{itemize}

\item Generalized causal model coefficients at both stages: normal distributions.
\item Added-on selection model coefficients: normal distributions.

\end{itemize}


\end{enumerate}

\end{Details}
%
\begin{Value}
If \emph{codaPkg=FALSE}(default), returns an object containing summary statistics of
the saved parameters, including
\begin{ldescription}
\item[\code{s1.intercept}] Estimate of the intercept from the first stage.
\item[\code{s1.slopeP}] Estimate of the pth slope from the first stage. 
\item[\code{s2.intercept}] Estimate of the intercept from the second stage.
\item[\code{s2.slopeP}] Estimate of the pth slope from the second stage (the first slope is always
the \strong{LATE}).
\item[\code{select.intercept}] Estimate of the intercept from the added-on selection model.
\item[\code{select.slope}] Estimate of the slope from the added-on selection model.
\item[\code{var.e.s2}] Estimate of the residual variance at the second stage.
\item[\code{DIC}] Deviance Information Criterion.
\end{ldescription}
If \emph{codaPkg=TRUE}, the returned value is the path for the output file
containing the Markov chain Monte Carlo output.
\end{Value}
%
\begin{References}\relax
Gelman, A., Carlin, J.B., Stern, H.S., Rubin, D.B. (2003).
\emph{Bayesian data analysis}, 2nd edition. Chapman and Hall/CRC Press.

Spiegelhalter, D. J., Thomas, A., Best, N. G., Gilks, W., \& Lunn, D. (1996).
BUGS: Bayesian inference using Gibbs sampling.
\Rhref{http://www.mrc-bsu.cam.ac.uk/bugs}{}
\end{References}
%
\begin{Examples}
\begin{ExampleCode}

# Run the model
model1 <- gts.nnormal.s(outcome~treatment|instrument,data=simCatNormMNAR,
m.ind=simCatNormMNAR$indicator,n.iter=100000)

# Run the model with the self-defined advanced feature
my.normal.s.model<- function(){
  for (i in 1:N){
    logit(p[i]) <- beta0 + beta1*z[i]
    x[i] ~ dbern(p[i])
    muY[i] <- gamma0 + gamma1*p[i]
    y[i] ~ dnorm(muY[i], pre.u2)

    m[i] ~ dbern(q[i])
    q[i] <- phi(lambda0 + lambda1*y[i])
  }

  beta0 ~ dnorm(0,1)
  beta1 ~ dnorm(1, 1)
  gamma0 ~ dnorm(0, 1)
  gamma1 ~ dnorm(.5, 1)
  lambda0 ~ dnorm(0, 1.0E-6)
  lambda1 ~ dnorm(0, 1.0E-6)

  pre.u2 ~ dgamma(.001, .001)

  s1.intercept <- beta0
  s1.slope1 <- beta1
  s2.intercept <- gamma0
  s2.slope1 <- gamma1
  select.intercept <- lambda0
  select.slope <- lambda1
  var.e.s2 <- 1/pre.u2
}

model2 <- gts.nnormal.s(outcome~treatment|instrument,data=simCatNormMNAR,
m.ind=simCatNormMNAR$indicator, advanced=TRUE, adv.model=my.normal.s.model,n.iter=100000)

# Extract the model DIC
model1$DIC

# Extract the MCMC output
model3 <- gts.nnormal.s(outcome~treatment|instrument,data=simCatNormMNAR,
m.ind=simCatNormMNAR$indicator,n.iter=100000,codaPkg=TRUE)


\end{ExampleCode}
\end{Examples}
\inputencoding{utf8}
\HeaderA{gts.nrobust}{Apply the generalized Bayesian two-stage robust-based causal model with instrumental variables.}{gts.nrobust}
%
\begin{Description}\relax
The \code{gts.nrobust} function applies the generalized Bayesian two-stage
robust-based causal model to the categorical treatment data.
The model best suits the outcome data that contain outliers
and are ignorably missing (i.e., MCAR or MAR).
\end{Description}
%
\begin{Usage}
\begin{verbatim}
gts.nrobust(formula, data, advanced = FALSE, adv.model, b0 = 1,
  B0 = 1e-06, g0 = 0, G0 = 1e-06, e0 = 0.001, E0 = 0.001,
  v0 = 0, V0 = 100, beta.start = NULL, gamma.start = NULL,
  e.start = NULL, df.start = 5, n.chains = 1,
  n.burnin = floor(n.iter/2), n.iter = 10000, n.thin = 1, DIC,
  debug = FALSE, codaPkg = FALSE)
\end{verbatim}
\end{Usage}
%
\begin{Arguments}
\begin{ldescription}
\item[\code{formula}] An object of class formula: a symbolic description of the model to be fitted.
The details of the model specification are given under "Details".

\item[\code{data}] A dataframe with the variables to be used in the model.

\item[\code{advanced}] Logical; if FALSE (default), the model is specified using the formula argument,
if TRUE, self-defined models can be specified using the adv.model argument.

\item[\code{adv.model}] Specify the self-defined model. Used when advanced=TRUE.

\item[\code{b0}] The mean hyperparameter of the normal distribution (prior distribution)
for the first-stage generalized causal model coefficients, i.e., coefficients for the instrumental variables.
This can either be a numerical value or a vector with dimensions equal to the number of coefficients
for the instrumental variables. If this takes a numerical value, then that values will
serve as the mean hyperparameter for all of the coefficients for the instrumental variables.
Default value of 0 is equivalent to a noninformative prior for the normal distributions.
Used when advanced=FALSE.

\item[\code{B0}] The precision hyperparameter of the normal distribution (prior distribution)
for the first stage generalized causal model coefficients.
This can either be a numerical value or a vector with dimensions equal to the number of coefficients
for the instrumental variables. If this takes a numerical value, then that values will
serve as the precision hyperparameter for all of the coefficients for the instrumental variables.
Default value of 1.0E-6 is equivalent to a noninformative prior for the normal distributions.
Used when advanced=FALSE.

\item[\code{g0}] The mean hyperparameter of the normal distribution (prior distribution)
for the second-stage generalized causal model coefficients,
i.e., coefficients for the treatment variable and other regression covariates.
This can either be a numerical value if there is only one treatment variable in the model,
or a if there is a treatment variable and multiple regression covariates,
with dimensions equal to the total number of coefficients for the treatment variable and covariates.
Default value of 0 is equivalent to a noninformative prior for the normal distributions.
Used when advanced=FALSE.

\item[\code{G0}] The precision hyperparameter of the normal distribution (prior distribution)
for the second-stage generalized causal model coefficients.
This can either be a numerical value if there is only one treatment variable in the model,
or a vector if there is a treatment variable and multiple regression covariates,
with dimensions equal to the total number of coefficients for the treatment variable and covariates.
Default value of 1.0E-6 is equivalent to a noninformative prior for the normal distributions.
Used when advanced=FALSE.

\item[\code{e0}] The location hyperparameter of the inverse Gamma distribution (prior for the scale parameter
of Student's t distribution on the model residual).
Default of 0.001 is equivalent to the noninformative prior for the inverse Gamma distribution.

\item[\code{E0}] The shape hyperparameter of the inverse Gamma distribution (prior for the scale parameter
of Student's t distribution on the model residual).
Default of 0.001 is equivalent to the noninformative prior for the inverse Gamma distribution.

\item[\code{v0}] The lower boundary hyperparameter of the uniform distribution (prior for the degrees of freedom
parameter of Student's t distribution).

\item[\code{V0}] The upper boundary hyperparameter of the uniform distribution (prior for the degrees of freedom
parameter of Student's t distribution).

\item[\code{beta.start}] The starting values for the first-stage generalized causal model coefficients,
i.e., coefficients for the instrumental variables.
This can either be a numerical value or a column vector with dimensions
equal to the number of first-stage coefficients.
The default value of NA will use the IWLS (iteratively reweighted least squares) estimate
of first-stage coefficients as the starting value.
If this is a numerical value, that value will
serve as the starting value mean for all the first-stage beta coefficients.

\item[\code{gamma.start}] The starting values for the second-stage generalized causal model coefficients,
i.e., coefficients for the treatment variable and the model covariates.
This can either be a numerical value or a column vector with dimensions
equal to the number of second-stage coefficients.
The default value of NA will use the IWLS (iteratively reweighted least squares) estimate
of second-stage coefficients as the starting value.
If this is a numerical value, that value will
serve as the starting value mean for all the second-stage gamma coefficients.

\item[\code{e.start}] The starting value for the precision hyperparameter of the inverse gamma distribution
(prior for the scale parameter of Student's t distribution of the model residual).
The default value of NA will use the inverse of the residual variance from the
IWLS (iteratively reweighted least square) estimate of the second-stage model.

\item[\code{df.start}] The starting value for the degrees of freedom of Student's t distribution.

\item[\code{n.chains}] The number of Markov chains. The default is 1.

\item[\code{n.burnin}] Length of burn in, i.e., number of iterations to discard at the beginning.
Default is n.iter/2, that is, discarding the first half of the simulations.

\item[\code{n.iter}] The number of total iterations per chain (including burnin). The default is 10000.

\item[\code{n.thin}] The thinning rate. Must be a positive integer. The default is 1.

\item[\code{DIC}] Logical; if TRUE (default), compute deviance, pD, and DIC. The rule pD=Dbar-Dhat is used.

\item[\code{codaPkg}] Logical; if FALSE (default), an object is returned; if TRUE,
file names of the output are returned.
\end{ldescription}
\end{Arguments}
%
\begin{Details}\relax
\begin{enumerate}

\item The formula takes the form \emph{response \textasciitilde{} terms|instrumental\_variables}.
\code{\LinkA{gts.nnormal}{gts.nnormal}} provides a detailed description of the formula rule.
\item DIC is computed as \emph{mean(deviance)+pD}.
\item Prior distributions used in ALMOND.
\begin{itemize}

\item Generalized causal model coefficients at both stages: normal distributions.
\item The generalized causal model residual: Student's t distribution.

\end{itemize}


\end{enumerate}

\end{Details}
%
\begin{Value}
If \emph{codaPkg=FALSE}(default), returns an object containing summary statistics of
the saved parameters, including
\begin{ldescription}
\item[\code{s1.intercept}] Estimate of the intercept from the first stage.
\item[\code{s1.slopeP}] Estimate of the pth slope from the first stage. 
\item[\code{s2.intercept}] Estimate of the intercept from the second stage.
\item[\code{s2.slopeP}] Estimate of the pth slope from the second stage (the first slope is always
the \strong{LATE}).
\item[\code{var.e.s2}] Estimate of the residual variance at the second stage.
\item[\code{df.est}] Estimate of the degrees of freedom for the Student's t distribution.
\item[\code{DIC}] Deviance Information Criterion.
\end{ldescription}
If \emph{codaPkg=TRUE}, the returned value is the path for the output file
containing the Markov chain Monte Carlo output.
\end{Value}
%
\begin{References}\relax
Gelman, A., Carlin, J.B., Stern, H.S., Rubin, D.B. (2003).
\emph{Bayesian data analysis}, 2nd edition. Chapman and Hall/CRC Press.

Spiegelhalter, D. J., Thomas, A., Best, N. G., Gilks, W., \& Lunn, D. (1996).
BUGS: Bayesian inference using Gibbs sampling.
\Rhref{http://www.mrc-bsu.cam.ac.uk/bugs}{}
\end{References}
%
\begin{Examples}
\begin{ExampleCode}

# Run the model
model1 <- gts.nrobust(neighborhoodRating~voucherProgram|extraBedroom,data=simVoucher)

# Run the model with the self-defined advanced feature
my.robust.model<- function(){
  for (i in 1:N){
    logit(p[i]) <- beta0 + beta1*z[i]
    x[i] ~ dbern(p[i])
    muY[i] <- gamma0 + gamma1*p[i]
    y[i] ~ dt(muY[i], pre.u2, df)
  }

  beta0 ~ dnorm(0,1)
  beta1 ~ dnorm(1, 1)
  gamma0 ~ dnorm(0, 1)
  gamma1 ~ dnorm(.5, 1)

  pre.u2 ~ dgamma(.001, .001)

  df ~ dunif(0,100)

  s1.intercept <- beta0
  s1.slope1 <- beta1
  s2.intercept <- gamma0
  s2.slope1 <- gamma1
  df.est <- df
  var.e.s2 <- 1/pre.u2
}

model2 <- gts.nrobust(neighborhoodRating~voucherProgram|extraBedroom,data=simVoucher,
advanced=TRUE, adv.model=my.robust.model)

# Extract the model DIC
model1$DIC

# Extract the MCMC output
model3 <- gts.nrobust(neighborhoodRating~voucherProgram|extraBedroom,data=simVoucher,
codaPkg=TRUE)


\end{ExampleCode}
\end{Examples}
\inputencoding{utf8}
\HeaderA{gts.nrobust.s}{Apply the generalized Bayesian two-stage robust-selection causal model with instrumental variables.}{gts.nrobust.s}
%
\begin{Description}\relax
The \code{gts.nrobust.s} function applies the generalized Bayesian two-stage
robust-selection causal model to the categorical treatment data.
The model best suits the outcome data that contain outliers
and are nonignorably missing (i.e., MNAR) (e.g., dropout, attrition).
\end{Description}
%
\begin{Usage}
\begin{verbatim}
gts.nrobust.s(formula, data, m.ind, advanced = FALSE, adv.model,
  b0 = 1, B0 = 1e-06, g0 = 0, G0 = 1e-06, e0 = 0.001,
  E0 = 0.001, l0 = 0, L0 = 1e-06, v0 = 0, V0 = 100,
  beta.start = NULL, gamma.start = NULL, e.start = NULL,
  lambda0.start = 1, lambda1.start = 1, df.start = 5, n.chains = 1,
  n.burnin = floor(n.iter/2), n.iter = 50000, n.thin = 1, DIC,
  debug = FALSE, codaPkg = FALSE)
\end{verbatim}
\end{Usage}
%
\begin{Arguments}
\begin{ldescription}
\item[\code{formula}] An object of class formula: a symbolic description of the model to be fitted.
The details of the model specification are given under "Details".

\item[\code{data}] A dataframe with the variables to be used in the model.

\item[\code{advanced}] Logical; if FALSE (default), the model is specified using the formula argument,
if TRUE, self-defined models can be specified using the adv.model argument.

\item[\code{adv.model}] Specify the self-defined model. Used when advanced=TRUE.

\item[\code{b0}] The mean hyperparameter of the normal distribution (prior distribution)
for the first-stage generalized causal model coefficients, i.e., coefficients for the instrumental variables.
This can either be a numerical value or a vector with dimensions equal to the number of coefficients
for the instrumental variables. If this takes a numerical value, then that values will
serve as the mean hyperparameter for all of the coefficients for the instrumental variables.
Default value of 0 is equivalent to a noninformative prior for the normal distributions.
Used when advanced=FALSE.

\item[\code{B0}] The precision hyperparameter of the normal distribution (prior distribution)
for the first stage generalized causal model coefficients.
This can either be a numerical value or a vector with dimensions equal to the number of coefficients
for the instrumental variables. If this takes a numerical value, then that values will
serve as the precision hyperparameter for all of the coefficients for the instrumental variables.
Default value of 1.0E-6 is equivalent to a noninformative prior for the normal distributions.
Used when advanced=FALSE.

\item[\code{g0}] The mean hyperparameter of the normal distribution (prior distribution)
for the second-stage generalized causal model coefficients,
i.e., coefficients for the treatment variable and other regression covariates.
This can either be a numerical value if there is only one treatment variable in the model,
or a if there is a treatment variable and multiple regression covariates,
with dimensions equal to the total number of coefficients for the treatment variable and covariates.
Default value of 0 is equivalent to a noninformative prior for the normal distributions.
Used when advanced=FALSE.

\item[\code{G0}] The precision hyperparameter of the normal distribution (prior distribution)
for the second-stage generalized causal model coefficients.
This can either be a numerical value if there is only one treatment variable in the model,
or a vector if there is a treatment variable and multiple regression covariates,
with dimensions equal to the total number of coefficients for the treatment variable and covariates.
Default value of 1.0E-6 is equivalent to a noninformative prior for the normal distributions.
Used when advanced=FALSE.

\item[\code{e0}] The location hyperparameter of the inverse Gamma distribution (prior for the scale parameter
of Student's t distribution on the model residual).
Default of 0.001 is equivalent to the noninformative prior for the inverse Gamma distribution.

\item[\code{E0}] The shape hyperparameter of the inverse Gamma distribution (prior for the scale parameter
of Student's t distribution on the model residual).
Default of 0.001 is equivalent to the noninformative prior for the inverse Gamma distribution.

\item[\code{l0}] The mean hyperparameter of the normal distribution (prior for the added-on selection model coefficients).
This can either be a numerical value or a vector with dimensions equal to the number of coefficients for the instrumental variables.
If this takes a numerical value, then that values will serve as the mean hyperparameter for all of the coefficients
for the instrumental variables. Default value of 0 is equivalent to a noninformative prior for the normal distributions.
Used when advanced=FALSE.

\item[\code{L0}] The precision hyperparameter of the normal distribution (prior for the added-on selection model coefficients).
This can either be a numerical value or a vector with dimensions equal to the number of coefficients for the instrumental variables.
If this takes a numerical value, then that values will serve as the precision hyperparameter for all of the coefficients
for the instrumental variables.
Default value of 1.0E-6 is equivalent to a noninformative prior for the normal distributions.
Used when advanced=FALSE.

\item[\code{v0}] The lower boundary hyperparameter of the uniform distribution (prior for the degrees of freedom
parameter of Student's t distribution).

\item[\code{V0}] The upper boundary hyperparameter of the uniform distribution (prior for the degrees of freedom
parameter of Student's t distribution).

\item[\code{beta.start}] The starting values for the first-stage generalized causal model coefficients,
i.e., coefficients for the instrumental variables.
This can either be a numerical value or a column vector with dimensions
equal to the number of first-stage coefficients.
The default value of NA will use the IWLS (iteratively reweighted least squares) estimate
of first-stage coefficients as the starting value.
If this is a numerical value, that value will
serve as the starting value mean for all the first-stage beta coefficients.

\item[\code{gamma.start}] The starting values for the second-stage generalized causal model coefficients,
i.e., coefficients for the treatment variable and the model covariates.
This can either be a numerical value or a column vector with dimensions
equal to the number of second-stage coefficients.
The default value of NA will use the IWLS (iteratively reweighted least squares) estimate
of second-stage coefficients as the starting value.
If this is a numerical value, that value will
serve as the starting value mean for all the second-stage gamma coefficients.

\item[\code{e.start}] The starting value for the precision hyperparameter of the inverse gamma distribution
(prior for the scale parameter of Student's t distribution of the model residual).
The default value of NA will use the inverse of the residual variance from the
IWLS (iteratively reweighted least square) estimate of the second-stage model.

\item[\code{lambda0.start}] The starting value for the intercept of the coefficient of the added-on selection model.

\item[\code{lambda1.start}] The starting value for the slope of the coefficient of the added-on selection model.

\item[\code{df.start}] The starting value for the degrees of freedom of Student's t distribution.

\item[\code{n.chains}] The number of Markov chains. The default is 1.

\item[\code{n.burnin}] Length of burn in, i.e., number of iterations to discard at the beginning.
Default is n.iter/2, that is, discarding the first half of the simulations.

\item[\code{n.iter}] The number of total iterations per chain (including burnin). The default is 50000.

\item[\code{n.thin}] The thinning rate. Must be a positive integer. The default is 1.

\item[\code{DIC}] Logical; if TRUE (default), compute deviance, pD, and DIC. The rule pD=Dbar-Dhat is used.

\item[\code{codaPkg}] Logical; if FALSE (default), an object is returned; if TRUE,
file names of the output are returned.
\end{ldescription}
\end{Arguments}
%
\begin{Details}\relax
\begin{enumerate}

\item The formula takes the form \emph{response \textasciitilde{} terms|instrumental\_variables}.
\code{\LinkA{gts.nnormal}{gts.nnormal}} provides a detailed description of the formula rule.
\item DIC is computed as \emph{mean(deviance)+pD}.
\item Prior distributions used in ALMOND.
\begin{itemize}

\item Generalized causal model coefficients at both stages: normal distributions.
\item The generalized causal model residual: Student's t distribution.
\item Added-on selection model coefficients: normal distributions.

\end{itemize}


\end{enumerate}

\end{Details}
%
\begin{Value}
If \emph{codaPkg=FALSE}(default), returns an object containing summary statistics of
the saved parameters, including
\begin{ldescription}
\item[\code{s1.intercept}] Estimate of the intercept from the first stage.
\item[\code{s1.slopeP}] Estimate of the pth slope from the first stage. 
\item[\code{s2.intercept}] Estimate of the intercept from the second stage.
\item[\code{s2.slopeP}] Estimate of the pth slope from the second stage (the first slope is always
the \strong{LATE}).
\item[\code{select.intercept}] Estimate of the intercept from the added-on selection model.
\item[\code{select.slope}] Estimate of the slope from the added-on selection model.
\item[\code{var.e.s2}] Estimate of the residual variance at the second stage.
\item[\code{df.est}] Estimate of the degrees of freedom for the Student's t distribution.
\item[\code{DIC}] Deviance Information Criterion.
\end{ldescription}
If \emph{codaPkg=TRUE}, the returned value is the path for the output file
containing the Markov chain Monte Carlo output.
\end{Value}
%
\begin{References}\relax
Gelman, A., Carlin, J.B., Stern, H.S., Rubin, D.B. (2003).
\emph{Bayesian data analysis}, 2nd edition. Chapman and Hall/CRC Press.

Spiegelhalter, D. J., Thomas, A., Best, N. G., Gilks, W., \& Lunn, D. (1996).
BUGS: Bayesian inference using Gibbs sampling.
\Rhref{http://www.mrc-bsu.cam.ac.uk/bugs}{}
\end{References}
%
\begin{Examples}
\begin{ExampleCode}

# Run the model
model1 <- gts.nrobust.s(outcome~treatment|instrument,data=simCatOutMNAR,
m.ind=simCatOutMNAR$indicator,n.iter=100000)

# Run the model with the self-defined advanced feature
my.model <- function(){
  for (i in 1:N){
    logit(p[i]) <- beta0 + beta1*z[i]
    x[i] ~ dbern(p[i])
    muY[i] <- gamma0 + gamma1*p[i]
    y[i] ~ dt(muY[i], pre.u2, df)

    m[i] ~ dbern(q[i])
    q[i] <- phi(lambda0 + lambda1*y[i])
  }

  beta0 ~ dnorm(0,1)
  beta1 ~ dnorm(1, 1)
  gamma0 ~ dnorm(0, 1)
  gamma1 ~ dnorm(.5, 1)
  lambda0 ~ dnorm(0, 1.0E-6)
  lambda1 ~ dnorm(0, 1.0E-6)

  pre.u2 ~ dgamma(.001, .001)

  df ~ dunif(0,50)

  s1.intercept <- beta0
  s1.slope1 <- beta1
  s2.intercept <- gamma0
  s2.slope1 <- LATE
  select.intercept <- lambda0
  select.slope <- lambda1
  df.est <- df
  var.e.s2 <- 1/pre.u2
}

model2 <- ts.nrobust.s(outcome~treatment|instrument,m.ind=simCatOutMNAR$indicator,data=simCatOutMNAR,
advanced=TRUE, adv.model=my.model,n.iter=100000)

# Extract the model DIC
model1$DIC

# Extract the MCMC output
model3 <- gts.nrobust.s(outcome~treatment|instrument,data=simCatOutMNAR,
m.ind=simCatOutMNAR$indicator,n.iter=100000, codaPkg=TRUE)


\end{ExampleCode}
\end{Examples}
\inputencoding{utf8}
\HeaderA{simCatNormMCAR}{A simulated dataset 1.}{simCatNormMCAR}
\keyword{datasets}{simCatNormMCAR}
%
\begin{Description}\relax
A simulated dataset with the categorical treatment variable. The outcome variable is
normally-distributed and is missing completely at random (MCAR).
The dataset contains a data frame with 600 rows (participants) and 4 columns (variables).
The variables are as follows.
\end{Description}
%
\begin{Usage}
\begin{verbatim}
data(simCatNormMCAR)
\end{verbatim}
\end{Usage}
%
\begin{Format}
A data frame with 600 rows and 4 columns .
\end{Format}
%
\begin{Details}\relax
\begin{itemize}

\item outcome. The hypothetical causal outcome variable.
\item treatment. The hypothetical causal treatment variable.
\item instrument. The hypothetical instrumental variable.
\item mis.ind. Is the outcome variable value missing? 1=Yes, 0=No.

\end{itemize}

\end{Details}
%
\begin{Examples}
\begin{ExampleCode}

data(simCatNormMCAR)


\end{ExampleCode}
\end{Examples}
\inputencoding{utf8}
\HeaderA{simCatNormMNAR}{A simulated dataset 2.}{simCatNormMNAR}
\keyword{datasets}{simCatNormMNAR}
%
\begin{Description}\relax
A simulated dataset with the categorical treatment variable. The outcome variable
is normally-distributed and is missing not at random (MNAR, e.g., dropout or attrition).
The dataset contains a data frame with 600 rows (participants) and 4 columns (variables).
The variables are as follows.
\end{Description}
%
\begin{Usage}
\begin{verbatim}
data(simCatNormMNAR)
\end{verbatim}
\end{Usage}
%
\begin{Format}
A data frame with 600 rows and 4 columns .
\end{Format}
%
\begin{Details}\relax
\begin{itemize}

\item outcome. The hypothetical causal outcome variable.
\item treatment. The hypothetical causal treatment variable.
\item instrument. The hypothetical instrumental variable.
\item mis.ind. Is the outcome variable value missing? 1=Yes, 0=No.

\end{itemize}

\end{Details}
%
\begin{Examples}
\begin{ExampleCode}

data(simCatNormMNAR)


\end{ExampleCode}
\end{Examples}
\inputencoding{utf8}
\HeaderA{simCatOutMCAR}{A simulated dataset 3.}{simCatOutMCAR}
\keyword{datasets}{simCatOutMCAR}
%
\begin{Description}\relax
A simulated dataset with the categorical treatment variables. The outcome variable contains
outliers and is missing completely at random (MCAR).
The dataset contains a data frame with 600 rows (participants) and 4 columns (variables).
The variables are as follows.
\end{Description}
%
\begin{Usage}
\begin{verbatim}
data(simCatOutMCAR)
\end{verbatim}
\end{Usage}
%
\begin{Format}
A data frame with 600 rows and 4 columns .
\end{Format}
%
\begin{Details}\relax
\begin{itemize}

\item outcome. The hypothetical causal outcome variable.
\item treatment. The hypothetical causal treatment variable.
\item instrument. The hypothetical instrumental variable.
\item mis.ind. Is the outcome variable value missing? 1=Yes, 0=No.

\end{itemize}

\end{Details}
%
\begin{Examples}
\begin{ExampleCode}

data(simCatOutMCAR)


\end{ExampleCode}
\end{Examples}
\inputencoding{utf8}
\HeaderA{simCatOutMNAR}{A simulated dataset 4.}{simCatOutMNAR}
\keyword{datasets}{simCatOutMNAR}
%
\begin{Description}\relax
A simulated dataset with the categorical treatment variable. The outcome variable contains
outliers and is missing not at random (MNAR, e.g., dropout or attrition).
The dataset contains a data frame with 600 rows (participants) and 4 columns (variables).
The variables are as follows.
\end{Description}
%
\begin{Usage}
\begin{verbatim}
data(simCatOutMNAR)
\end{verbatim}
\end{Usage}
%
\begin{Format}
A data frame with 600 rows and 4 columns .
\end{Format}
%
\begin{Details}\relax
\begin{itemize}

\item outcome. The hypothetical causal outcome variable.
\item treatment. The hypothetical causal treatment variable.
\item instrument. The hypothetical instrumental variable.
\item mis.ind. Is the outcome variable value missing? 1=Yes, 0=No.

\end{itemize}

\end{Details}
%
\begin{Examples}
\begin{ExampleCode}

data(simCatOutMNAR)


\end{ExampleCode}
\end{Examples}
\inputencoding{utf8}
\HeaderA{simNormMCAR}{A simulated dataset 5.}{simNormMCAR}
\keyword{datasets}{simNormMCAR}
%
\begin{Description}\relax
A simulated dataset with the continous treatment variable. The outcome variable is
normally-distributed and is missing completely at random (MCAR).
The dataset contains a data frame with 600 rows (participants) and 4 columns (variables).
The variables are as follows.
\end{Description}
%
\begin{Usage}
\begin{verbatim}
data(simNormMCAR)
\end{verbatim}
\end{Usage}
%
\begin{Format}
A data frame with 600 rows and 4 columns .
\end{Format}
%
\begin{Details}\relax
\begin{itemize}

\item outcome. The hypothetical causal outcome variable.
\item treatment. The hypothetical causal treatment variable.
\item instrument. The hypothetical instrumental variable.
\item mis.ind. Is the outcome variable value missing? 1=Yes, 0=No.

\end{itemize}

\end{Details}
%
\begin{Examples}
\begin{ExampleCode}

data(simNormMCAR)


\end{ExampleCode}
\end{Examples}
\inputencoding{utf8}
\HeaderA{simNormMNAR}{A simulated dataset 6.}{simNormMNAR}
\keyword{datasets}{simNormMNAR}
%
\begin{Description}\relax
A simulated dataset with the continuous treatment variable. The outcome variable
is normally-distributed and is missing not at random (MNAR, e.g., dropout or attrition).
The dataset contains a data frame with 600 rows (participants) and 4 columns (variables).
The variables are as follows.
\end{Description}
%
\begin{Usage}
\begin{verbatim}
data(simNormMNAR)
\end{verbatim}
\end{Usage}
%
\begin{Format}
A data frame with 600 rows and 4 columns .
\end{Format}
%
\begin{Details}\relax
\begin{itemize}

\item outcome. The hypothetical causal outcome variable.
\item treatment. The hypothetical causal treatment variable.
\item instrument. The hypothetical instrumental variable.
\item mis.ind. Is the outcome variable value missing? 1=Yes, 0=No.

\end{itemize}

\end{Details}
%
\begin{Examples}
\begin{ExampleCode}

data(simNormMNAR)


\end{ExampleCode}
\end{Examples}
\inputencoding{utf8}
\HeaderA{simOutMCAR}{A simulated dataset 7.}{simOutMCAR}
\keyword{datasets}{simOutMCAR}
%
\begin{Description}\relax
A simulated dataset with the continuous treatment variables. The outcome variable contains
outliers and is missing completely at random (MCAR).
The dataset contains a data frame with 600 rows (participants) and 4 columns (variables).
The variables are as follows.
\end{Description}
%
\begin{Usage}
\begin{verbatim}
data(simOutMCAR)
\end{verbatim}
\end{Usage}
%
\begin{Format}
A data frame with 600 rows and 4 columns .
\end{Format}
%
\begin{Details}\relax
\begin{itemize}

\item outcome. The hypothetical causal outcome variable.
\item treatment. The hypothetical causal treatment variable.
\item instrument. The hypothetical instrumental variable.
\item mis.ind. Is the outcome variable value missing? 1=Yes, 0=No.

\end{itemize}

\end{Details}
%
\begin{Examples}
\begin{ExampleCode}

data(simOutMCAR)


\end{ExampleCode}
\end{Examples}
\inputencoding{utf8}
\HeaderA{simOutMNAR}{A simulated dataset 8.}{simOutMNAR}
\keyword{datasets}{simOutMNAR}
%
\begin{Description}\relax
A simulated dataset with the continuous treatment variable. The outcome variable contains
outliers and is missing not at random (MNAR, e.g., dropout or attrition).
The dataset contains a data frame with 600 rows (participants) and 4 columns (variables).
The variables are as follows.
\end{Description}
%
\begin{Usage}
\begin{verbatim}
data(simOutMNAR)
\end{verbatim}
\end{Usage}
%
\begin{Format}
A data frame with 600 rows and 4 columns .
\end{Format}
%
\begin{Details}\relax
\begin{itemize}

\item outcome. The hypothetical causal outcome variable.
\item treatment. The hypothetical causal treatment variable.
\item instrument. The hypothetical instrumental variable.
\item mis.ind. Is the outcome variable value missing? 1=Yes, 0=No.

\end{itemize}

\end{Details}
%
\begin{Examples}
\begin{ExampleCode}

data(simOutMNAR)


\end{ExampleCode}
\end{Examples}
\inputencoding{utf8}
\HeaderA{simVoucher}{Simulated data of a public voucher program.}{simVoucher}
\keyword{datasets}{simVoucher}
%
\begin{Description}\relax
The dataset is simulated from a study of the effect of a public voucher program
(Currie and Yelowitz, 2000). The dataset contains the status of the public voucher program
participation and other attributes. The variables are as follows.
\end{Description}
%
\begin{Usage}
\begin{verbatim}
data(simVoucher)
\end{verbatim}
\end{Usage}
%
\begin{Format}
A data frame with 1954 rows (participants) and 10 columns (variables).
\end{Format}
%
\begin{References}\relax
Currie, J., \& Yelowitz, A. (2000).
Are public housing projects good for kids? \emph{Journal of public economics},
\emph{75}(1), 99-124.
\end{References}
%
\begin{Examples}
\begin{ExampleCode}

data(simVoucher)


\end{ExampleCode}
\end{Examples}
\inputencoding{utf8}
\HeaderA{subECLSK}{A subset of the ECLSK (Early Childhood Longitudinal Study – Kindergarten) cohort data.}{subECLSK}
\keyword{datasets}{subECLSK}
%
\begin{Description}\relax
The dataset contains the reading and mathematics scores and other attributes of
600 kindergarteners. The variables are as follows.
\end{Description}
%
\begin{Usage}
\begin{verbatim}
data(subECLSK)
\end{verbatim}
\end{Usage}
%
\begin{Format}
A data frame with 600 rows (participants) and 11 columns (variables).
\end{Format}
%
\begin{Details}\relax
\begin{itemize}

\item dobmm. The date of birth measured in months.
\item gender. 1=Male; 2=Female
\item race.
\begin{itemize}

\item 1 = White,non-Hispanic
\item 2 = Black or African American, non-Hispanic
\item 3 = Hispanic, race specified
\item 4 = Hispanic, race no specified
\item 5 = Asian
\item 6 = Native Hawaiin, other Pacific islander
\item 7 = American Indian or Alaska native
\item 8 = More than one race, non-Hispanic

\end{itemize}

\item readingIRT. Reading IRT (Item Response Theory) scaled score.
\item mathIRT. Mathematics IRT (Item Response Theory) scaled score.
\item numsib. Number of siblings in household.
\item parentedu. Parent highest education level.
\item ses. Continuous socioeconomic status measure.
\item relAge. The relative age of the participant entering kindergarten measured in months.
\item PredEnt. The predicted relative kindergarten entrance age.
\item mis.ind.read. Is the reading IRT score missing? 1=Yes, 0=No.
\item mis.ind.math. Is the math IRT score missing? 1=Yes, 0=No.

\end{itemize}

\end{Details}
%
\begin{References}\relax
Zhong, S. Y., \& Hoxby, C. M. (2012).
The effects of relative age on early childhood academic achievement:
how they differ between gender and change across time.
(Unpublished Honors Thesis) Stanford University, Stanford, CA.
(\Rhref{http://economics. stanford. edu/_les/StephanieYuechenZhongHonorsThesis2012.}{}
Tourangeau, K., Nord, C., Lê, T., Sorongon, A. G., \& Najarian, M. (2009).
Early childhood longitudinal study, kindergarten class of 1998–99 (ECLS-K),
combined User's manual for the ECLS-K eighth-grade and K–8 full sample data files
and electronic codebooks (NCES2009–004).
National Center for Education Statistics, Institute of Education Sciences,
U.S. Departmentof Education. Washington DC.
\end{References}
%
\begin{Examples}
\begin{ExampleCode}

data(subECLSK)


\end{ExampleCode}
\end{Examples}
\inputencoding{utf8}
\HeaderA{ts.nnormal}{Apply the normal-based Bayesian two-stage causal model with instrumental variables.}{ts.nnormal}
%
\begin{Description}\relax
The \code{ts.nnormal} function applies the normal-based Bayesian two-stage causal model
to the continuous treatment data. The model best suits the normally-distributed outcome data
that are complete or ignorably missing (i.e., missing completely at random or missing at random).
\end{Description}
%
\begin{Usage}
\begin{verbatim}
ts.nnormal(formula, data, advanced = FALSE, adv.model, b0 = 1,
  B0 = 1e-06, g0 = 0, G0 = 1e-06, u0 = 0.001, U0 = 0.001,
  e0 = 0.001, E0 = 0.001, beta.start = NULL, gamma.start = NULL,
  u.start = NULL, e.start = NULL, n.chains = 1,
  n.burnin = floor(n.iter/2), n.iter = 10000, n.thin = 1, DIC,
  debug = FALSE, codaPkg = FALSE)
\end{verbatim}
\end{Usage}
%
\begin{Arguments}
\begin{ldescription}
\item[\code{b0}] The mean hyperparameter of the normal distribution (prior distribution)
for the first-stage causal model coefficients, i.e., coefficients for the instrumental variables.
This can either be a numerical value or a vector with dimensions equal to the number of coefficients
for the instrumental variables. If this takes a numerical value, then that values will
serve as the mean hyperparameter for all of the coefficients for the instrumental variables.
Default value of 0 is equivalent to a noninformative prior for the normal distributions.
Used when advanced=FALSE.

\item[\code{B0}] The precision hyperparameter of the normal distribution (prior distribution)
for the first stage causal model coefficients.
This can either be a numerical value or a vector with dimensions equal to the number of coefficients
for the instrumental variables. If this takes a numerical value, then that values will
serve as the precision hyperparameter for all of the coefficients for the instrumental variables.
Default value of 10E+6 is equivalent to a noninformative prior for the normal distributions.
Used when advanced=FALSE.

\item[\code{g0}] The mean hyperparameter of the normal distribution (prior distribution)
for the second-stage causal model coefficients,
i.e., coefficients for the treatment variable and other regression covariates).
This can either be a numerical value if there is only one treatment variable in the model,
or a if there is a treatment variable and multiple regression covariates,
with dimensions equal to the total number of coefficients for the treatment variable and covariates.
Default value of 0 is equivalent to a noninformative prior for the normal distributions.
Used when advanced=FALSE.

\item[\code{G0}] The precision hyperparameter of the normal distribution (prior distribution)
for the second-stage causal model coefficients.
This can either be a numerical value if there is only one treatment variable in the model,
or a vector if there is a treatment variable and multiple regression covariates,
with dimensions equal to the total number of coefficients for the treatment variable and covariates.
Default value of 10E+6 is equivalent to a noninformative prior for the normal distributions.
Used when advanced=FALSE.

\item[\code{u0}] The location hyperparameter of the inverse Gamma distribution (prior for the variance of the
normal distribution on the model residuals at the first stage).
Default of 0.001 is equivalent to the noninformative prior for the inverse Gamma distribution.

\item[\code{U0}] The shape hyperparameter of the inverse Gamma distribution (prior for the variance of the
normal distribution on the model residuals at the first stage).
Default of 0.001 is equivalent to the noninformative prior for the inverse Gamma distribution.

\item[\code{e0}] The location hyperparameter of the inverse Gamma distribution (prior for the variance of the
normal distribution on the model residuals at the second stage).
Default of 0.001 is equivalent to the noninformative prior for the inverse Gamma distribution.

\item[\code{E0}] The shape hyperparameter of the inverse Gamma distribution (prior for the the variance of the
normal distribution on the model residuals at the second stage).
Default of 0.001 is equivalent to the noninformative prior for the inverse Gamma distribution.

\item[\code{beta.start}] The starting values for the first-stage causal model coefficients,
i.e., coefficients for the instrumental variables.
This can either be a numerical value or a column vector with dimensions
equal to the number of first-stage coefficients.
The default value of NA will use the OLS estimate of first-stage coefficients as the starting value.
If this is a numerical value, that value will
serve as the starting value mean for all the first-stage beta coefficients.

\item[\code{gamma.start}] The starting values for the second-stage causal model coefficients,
i.e., coefficients for the treatment variable and the model covariates.
This can either be a numerical value or a column vector with dimensions
equal to the number of second-stage coefficients.
The default value of NA will use the OLS estimate of second-stage coefficients as the starting value.
If this is a numerical value, that value will
serve as the starting value mean for all the second-stage gamma coefficients.

\item[\code{u.start}] The starting value for the precision hyperparameter of the inverse gamma distribution
(prior for the variance of the normal distribution of the first-stage residual term).
The default value of NA will use the inverse of the residual variance from the OLS estimate of the first-stage model.

\item[\code{e.start}] The starting value for the precision hyperparameter of the inverse gamma distribution
(prior for the variance of the normal distribution of the second-stage residual term).
The default value of NA will use the inverse of the residual variance from the OLS estimate
of the second-stage model.

\item[\code{n.chains}] Number of Markov chains. The default is 1.

\item[\code{n.burnin}] Length of burn in, i.e., number of iterations to discard at the beginning.
Default is n.iter/2, that is, discarding the first half of the simulations.

\item[\code{n.iter}] The number of total iterations per chain (including burnin). The default is 10000.

\item[\code{n.thin}] Thinning rate. Must be a positive integer. The default is 1.

\item[\code{DIC}] Logical; if TRUE (default), compute deviance, pD, and DIC. The rule pD=Dbar-Dhat is used.

\item[\code{codaPkg}] Logical; if FALSE (default), an object is returned; if TRUE,
file names of the output are returned.
\end{ldescription}
\end{Arguments}
%
\begin{Details}\relax
\begin{enumerate}

\item Bayesian two-stage causal models are specified symbolically.
A typical model has the form \emph{reponse \textasciitilde{} terms|instrumental\_variables},
where response is the (numeric) response vector and terms is a series of terms
which specifies a linear predictor (i.e., the treatment variable and the covariates) for the response.
The first specification in the term is always the treatment variable and
the remaining specifications are always the covariates for the response.
\item DIC is computed as \emph{mean(deviance)+pD}.
\item Prior distributions used in ALMOND.
\begin{itemize}

\item Causal model coefficients at both stages: normal distributions.
\item Causal model residuals at both stages: normal distributions.

\end{itemize}


\end{enumerate}

\end{Details}
%
\begin{Value}
If \emph{codaPkg=FALSE}(default), returns an object containing summary statistics of
the saved parameters, including
\begin{ldescription}
\item[\code{s1.intercept}] Estimate of the intercept from the first stage.
\item[\code{s1.slopeP}] Estimate of the pth slope from the first stage. 
\item[\code{s2.intercept}] Estimate of the intercept from the second stage.
\item[\code{s2.slopeP}] Estimate of the pth slope from the second stage (the first slope is always
the \strong{LATE}).
\item[\code{var.e.s1}] Estimate of the residual variance at the first stage.
\item[\code{var.e.s2}] Estimate of the residual variance at the second stage.
\item[\code{DIC}] Deviance Information Criterion.
\end{ldescription}
If \emph{codaPkg=TRUE}, the returned value is the path for the output file
containing the Markov chain Monte Carlo output.
\end{Value}
%
\begin{Examples}
\begin{ExampleCode}

# Run the model
model1 <- ts.nnormal(readingIRT~relAge+gender+race+numsib+parentedu+ses|PredEnt,data=subECLSK)

# Run the model with the self-defined advanced feature
my.normal.model<- function(){
for (i in 1:N){
mu[i] <- beta0 + beta1*z[i]
x[i] ~ dnorm(mu[i], pre.u1)
muY[i] <- gamma0 + gamma1*mu[i]
y[i] ~ dnorm(muY[i], pre.u2)
}

beta0 ~ dnorm(0,1)
beta1 ~ dnorm(1, 1)
gamma0 ~ dnorm(0, 1)
gamma1 ~ dnorm(.5, 1)
pre.u1 ~ dgamma(.001, .001)
pre.u2 ~ dgamma(.001, .001)

s1.intercept <- beta0
s1.slope1 <- beta1
s2.intercept <- gamma0
s2.slope1 <- gamma1
var.e.s1 <- 1/pre.u1
var.e.s2 <- 1/pre.u2
}

model2 <- ts.nnormal(readingIRT~relAge+gender+race+numsib+parentedu+ses|PredEnt,data=subECLSK,
advanced=TRUE,adv.model=my.normal.model)

# Extract the model DIC
model1$DIC

# Extract the MCMC output
ts.nnormal(readingIRT~relAge+gender+race+numsib+parentedu+ses|PredEnt,data=subECLSK,codaPkg=TRUE)

\end{ExampleCode}
\end{Examples}
\inputencoding{utf8}
\HeaderA{ts.nnormal.s}{Apply the Bayesian two-stage normal-selection causal model with instrumental variables.}{ts.nnormal.s}
%
\begin{Description}\relax
The \code{ts.nnormal.s} function applies the Bayesian two-stage normal-selection causal model to the continuous treatment data.
The model best suits the outcome data that are normally-distributed and nonignorably missing
(i.e., MNAR) (e.g., dropout, attrition).
\end{Description}
%
\begin{Usage}
\begin{verbatim}
ts.nnormal.s(formula, data, m.ind, advanced = FALSE, adv.model, b0 = 0,
  B0 = 1e-06, g0 = 0, G0 = 1e-06, u0 = 0.001, U0 = 0.001,
  e0 = 0.001, E0 = 0.001, l0 = 0, L0 = 1e-06, beta.start = NULL,
  gamma.start = NULL, u.start = NULL, e.start = NULL,
  lambda0.start = 1, lambda1.start = 1, n.chains = 1,
  n.burnin = 5000, n.iter = 10000, n.thin = 1, DIC, debug = FALSE,
  codaPkg = FALSE)
\end{verbatim}
\end{Usage}
%
\begin{Arguments}
\begin{ldescription}
\item[\code{formula}] An object of class formula: a symbolic description of the model to be fitted.
The details of the model specification are given under "Details".

\item[\code{data}] A dataframe with the variables to be used in the model.

\item[\code{advanced}] Logical; if FALSE (default), the model is specified using the formula argument,
if TRUE, self-defined models can be specified using the adv.model argument.

\item[\code{adv.model}] Specify the self-defined model. Used when advanced=TRUE.

\item[\code{b0}] The mean hyperparameter of the normal distribution (prior distribution)
for the first-stage causal model coefficients, i.e., coefficients for the instrumental variables.
This can either be a numerical value or a vector with dimensions equal to the number of coefficients
for the instrumental variables. If this takes a numerical value, then that values will
serve as the mean hyperparameter for all of the coefficients for the instrumental variables.
Default value of 0 is equivalent to a noninformative prior for the normal distributions.
Used when advanced=FALSE.

\item[\code{B0}] The precision hyperparameter of the normal distribution (prior distribution)
for the first stage causal model coefficients.
This can either be a numerical value or a vector with dimensions equal to the number of coefficients
for the instrumental variables. If this takes a numerical value, then that values will
serve as the precision hyperparameter for all of the coefficients for the instrumental variables.
Default value of 10E+6 is equivalent to a noninformative prior for the normal distributions.
Used when advanced=FALSE.

\item[\code{g0}] The mean hyperparameter of the normal distribution (prior distribution)
for the second-stage causal model coefficients,
i.e., coefficients for the treatment variable and other regression covariates).
This can either be a numerical value if there is only one treatment variable in the model,
or a if there is a treatment variable and multiple regression covariates,
with dimensions equal to the total number of coefficients for the treatment variable and covariates.
Default value of 0 is equivalent to a noninformative prior for the normal distributions.
Used when advanced=FALSE.

\item[\code{G0}] The precision hyperparameter of the normal distribution (prior distribution)
for the second-stage causal model coefficients.
This can either be a numerical value if there is only one treatment variable in the model,
or a vector if there is a treatment variable and multiple regression covariates,
with dimensions equal to the total number of coefficients for the treatment variable and covariates.
Default value of 10E+6 is equivalent to a noninformative prior for the normal distributions.
Used when advanced=FALSE.

\item[\code{u0}] The location hyperparameter of the inverse Gamma distribution (prior for the variance of the
normal distribution on the model residuals at the first stage).
Default of 0.001 is equivalent to the noninformative prior for the inverse Gamma distribution.

\item[\code{U0}] The shape hyperparameter of the inverse Gamma distribution (prior for the variance of the
normal distribution on the model residuals at the first stage).
Default of 0.001 is equivalent to the noninformative prior for the inverse Gamma distribution.

\item[\code{e0}] The location hyperparameter of the inverse Gamma distribution (prior for the variance of the
normal distribution on the model residuals at the second stage).
Default of 0.001 is equivalent to the noninformative prior for the inverse Gamma distribution.

\item[\code{E0}] The shape hyperparameter of the inverse Gamma distribution (prior for the variance of the
normal distribution on the model residuals at the second stage).
Default of 0.001 is equivalent to the noninformative prior for the inverse Gamma distribution.

\item[\code{l0}] The mean hyperparameter of the normal distribution (prior for the added-on selection model coefficients).
This can either be a numerical value or a vector with dimensions equal to the number of coefficients for the instrumental variables.
If this takes a numerical value, then that values will serve as the mean hyperparameter for all of the coefficients
for the instrumental variables. Default value of 0 is equivalent to a noninformative prior for the normal distributions.
Used when advanced=FALSE.

\item[\code{L0}] The precision hyperparameter of the normal distribution (prior for the added-on selection model coefficients).
This can either be a numerical value or a vector with dimensions equal to the number of coefficients for the instrumental variables.
If this takes a numerical value, then that values will serve as the precision hyperparameter for all of the coefficients
for the instrumental variables. Default value of 10E+6 is equivalent to a noninformative prior for the normal distributions.
Used when advanced=FALSE.

\item[\code{beta.start}] The starting values for the first-stage causal model coefficients,
i.e., coefficients for the instrumental variables.
This can either be a numerical value or a column vector with dimensions
equal to the number of first-stage coefficients.
The default value of NA will use the OLS estimate of first-stage coefficients as the starting value.
If this is a numerical value, that value will
serve as the starting value mean for all the first-stage beta coefficients.

\item[\code{gamma.start}] The starting values for the second-stage causal model coefficients,
i.e., coefficients for the treatment variable and the model covariates.
This can either be a numerical value or a column vector with dimensions
equal to the number of second-stage coefficients.
The default value of NA will use the OLS estimate of second-stage coefficients as the starting value.
If this is a numerical value, that value will
serve as the starting value mean for all the second-stage gamma coefficients.

\item[\code{u.start}] The starting value for the precision hyperparameter of the inverse gamma distribution
(prior for the variance of the normal distribution of the first-stage residual term).
The default value of NA will use the inverse of the residual variance from the OLS estimate of the first-stage model.

\item[\code{e.start}] The starting value for the precision hyperparameter of the inverse gamma distribution
(prior for the scale parameter of Student's t distribution of the second-stage residual term).
The default value of NA will use the inverse of the residual variance from the OLS estimate
of the second-stage model.

\item[\code{lambda0.start}] The starting value for the intercept of the coefficient of the added-on selection model.

\item[\code{lambda1.start}] The starting value for the slope of the coefficient of the added-on selection model.

\item[\code{n.chains}] Number of Markov chains. The default is 1.

\item[\code{n.burnin}] Length of burn in, i.e., number of iterations to discard at the beginning.
Default is n.iter/2, that is, discarding the first half of the simulations.

\item[\code{n.iter}] Number of total iterations per chain (including burnin). The default is 50000.

\item[\code{n.thin}] Thinning rate. Must be a positive integer. The default is 1.

\item[\code{DIC}] Logical; if TRUE (default), compute deviance, pD, and DIC. The rule pD=Dbar-Dhat is used.

\item[\code{codaPkg}] Logical; if FALSE (default), an object is returned; if TRUE,
file names of the output are returned.
\end{ldescription}
\end{Arguments}
%
\begin{Details}\relax
\begin{enumerate}

\item The formula takes the form \emph{response \textasciitilde{} terms|instrumental\_variables}.
\code{\LinkA{ts.nnormal}{ts.nnormal}} provides a detailed description of the formula rule.
\item DIC is computed as \emph{mean(deviance)+pD}.
\item Prior distributions used in ALMOND.
\begin{itemize}

\item Causal model coefficients at both stages: normal distributions.
\item Causal model residuals at both stages: normal distributions.
\item Added-on selection model coefficients: normal distributions.

\end{itemize}


\end{enumerate}

\end{Details}
%
\begin{Value}
If \emph{codaPkg=FALSE}(default), returns an object containing summary statistics of
the saved parameters, including
\begin{ldescription}
\item[\code{s1.intercept}] Estimate of the intercept from the first stage.
\item[\code{s1.slopeP}] Estimate of the pth slope from the first stage. 
\item[\code{s2.intercept}] Estimate of the intercept from the second stage.
\item[\code{s2.slopeP}] Estimate of the pth slope from the second stage (the first slope is always
the \strong{LATE}).
\item[\code{select.intercept}] Estimate of the intercept from the added-on selection model.
\item[\code{select.slope}] Estimate of the slope from the added-on selection model.
\item[\code{var.e.s1}] Estimate of the residual variance at the first stage.
\item[\code{var.e.s2}] Estimate of the residual variance at the second stage.
\item[\code{DIC}] Deviance Information Criterion.
\end{ldescription}
If \emph{codaPkg=TRUE}, the returned value is the path for the output file
containing the Markov chain Monte Carlo output.
\end{Value}
%
\begin{Examples}
\begin{ExampleCode}

# Run the model
model1 <- ts.nnormal.s(readingIRT~relAge+gender+race+numsib+parentedu+ses|PredEnt,data=subECLSK,
m.ind=subECLSK$mis.ind.read, n.iter=100000)

# Run the model with the self-defined advanced feature
my.normal.s.model<- function(){
  for (i in 1:N){
    mu[i] <- beta0 + beta1*z[i]
    x[i] ~ dnorm(mu[i], pre.u1)
    muY[i] <- gamma0 + gamma1*mu[i]
    y[i] ~ dnorm(muY[i], pre.u2)

    m[i] ~ dbern(q[i])
    q[i] <- phi(lambda0 + lambda1*y[i])
  }

  beta0 ~ dnorm(0,1)
  beta1 ~ dnorm(1, 1)
  gamma0 ~ dnorm(0, 1)
  gamma1 ~ dnorm(.5, 1)
  lambda0 ~ dnorm(0, 1.0E-6)
  lambda1 ~ dnorm(0, 1.0E-6)

  pre.u1 ~ dgamma(.001, .001)
  pre.u2 ~ dgamma(.001, .001)

  s1.intercept <- beta0
  s1.slope1 <- beta1
  s2.intercept <- gamma0
  s2.slope1 <- gamma1
  select.intercept <- lambda0
  select.slope <- lambda1
  var.e.s1 <- 1/pre.u1
  var.e.s2 <- 1/pre.u2
}

model2 <- ts.nnormal.s(readingIRT~relAge+gender+race+numsib+parentedu+ses|PredEnt,
m.ind=subECLSK$mis.ind.read,data=subECLSK,advanced=TRUE, adv.model=my.normal.selection.model,
n.iter=100000)

# Extract the model DIC
model1$DIC

# Extract the MCMC output
ts.nnormal.s(readingIRT~relAge+gender+race+numsib+parentedu+ses|PredEnt,
m.ind=subECLSK$mis.ind.read,data=subECLSK,codaPkg=TRUE)

\end{ExampleCode}
\end{Examples}
\inputencoding{utf8}
\HeaderA{ts.nrobust}{Apply the robust-based Bayesian two-stage causal model with instrumental variables.}{ts.nrobust}
%
\begin{Description}\relax
The \code{ts.nrobust} function applies the robust-based Bayesian two-stage causal model
to the continuous treatment data. The model best suits the outcome data that contain outliers and
are complete or ignorably missing (i.e., missing completely at random or missing at random).
\end{Description}
%
\begin{Usage}
\begin{verbatim}
ts.nrobust(formula, data, advanced = FALSE, adv.model, b0 = 1,
  B0 = 1e-06, g0 = 0, G0 = 1e-06, u0 = 0.001, U0 = 0.001,
  e0 = 0.001, E0 = 0.001, v0 = 0, V0 = 100, beta.start = NULL,
  gamma.start = NULL, u.start = NULL, e.start = NULL, df.start = 5,
  n.chains = 1, n.burnin = floor(n.iter/2), n.iter = 10000,
  n.thin = 1, DIC, debug = FALSE, codaPkg = FALSE)
\end{verbatim}
\end{Usage}
%
\begin{Arguments}
\begin{ldescription}
\item[\code{formula}] An object of class formula: a symbolic description of the model to be fitted.
The details of the model specification are given under "Details".

\item[\code{data}] A dataframe with the variables to be used in the model.

\item[\code{advanced}] Logical; if FALSE (default), the model is specified using the formula argument,
if TRUE, self-defined models can be specified using the adv.model argument.

\item[\code{adv.model}] Specify the self-defined model. Used when advanced=TRUE.

\item[\code{b0}] The mean hyperparameter of the normal distribution (prior distribution)
for the first-stage causal model coefficients, i.e., coefficients for the instrumental variables.
This can either be a numerical value or a vector with dimensions equal to the number of coefficients
for the instrumental variables. If this takes a numerical value, then that values will
serve as the mean hyperparameter for all of the coefficients for the instrumental variables.
Default value of 0 is equivalent to a noninformative prior for the normal distributions.
Used when advanced=FALSE.

\item[\code{B0}] The precision hyperparameter of the normal distribution (prior distribution)
for the first stage causal model coefficients.
This can either be a numerical value or a vector with dimensions equal to the number of coefficients
for the instrumental variables. If this takes a numerical value, then that values will
serve as the precision hyperparameter for all of the coefficients for the instrumental variables.
Default value of 10E+6 is equivalent to a noninformative prior for the normal distributions.
Used when advanced=FALSE.

\item[\code{g0}] The mean hyperparameter of the normal distribution (prior distribution)
for the second-stage causal model coefficients,
i.e., coefficients for the treatment variable and other regression covariates).
This can either be a numerical value if there is only one treatment variable in the model,
or a if there is a treatment variable and multiple regression covariates,
with dimensions equal to the total number of coefficients for the treatment variable and covariates.
Default value of 0 is equivalent to a noninformative prior for the normal distributions.
Used when advanced=FALSE.

\item[\code{G0}] The precision hyperparameter of the normal distribution (prior distribution)
for the second-stage causal model coefficients.
This can either be a numerical value if there is only one treatment variable in the model,
or a vector if there is a treatment variable and multiple regression covariates,
with dimensions equal to the total number of coefficients for the treatment variable and covariates.
Default value of 10E+6 is equivalent to a noninformative prior for the normal distributions.
Used when advanced=FALSE.

\item[\code{u0}] The location hyperparameter of the inverse Gamma distribution (prior for the variance of the
normal distribution on the model residuals at the first stage).
Default of 0.001 is equivalent to the noninformative prior for the inverse Gamma distribution.

\item[\code{U0}] The shape hyperparameter of the inverse Gamma distribution (prior for the variance of the
normal distribution on the model residuals at the first stage).
Default of 0.001 is equivalent to the noninformative prior for the inverse Gamma distribution.

\item[\code{e0}] The location hyperparameter of the inverse Gamma distribution (prior for the scale parameter
of Student's t distribution on the model residuals at the second stage).
Default of 0.001 is equivalent to the noninformative prior for the inverse Gamma distribution.

\item[\code{E0}] The shape hyperparameter of the inverse Gamma distribution (prior for the scale parameter
of Student's t distribution on the model residuals at the second stage).
Default of 0.001 is equivalent to the noninformative prior for the inverse Gamma distribution.

\item[\code{v0}] The lower boundary hyperparameter of the uniform distribution (prior for the degrees of freedom
parameter of Student's t distribution).

\item[\code{V0}] The upper boundary hyperparameter of the uniform distribution (prior for the degrees of freedom
parameter of Student's t distribution).

\item[\code{beta.start}] The starting values for the first-stage causal model coefficients,
i.e., coefficients for the instrumental variables.
This can either be a numerical value or a column vector with dimensions
equal to the number of first-stage coefficients.
The default value of NA will use the OLS estimate of first-stage coefficients as the starting value.
If this is a numerical value, that value will
serve as the starting value mean for all the first-stage beta coefficients.

\item[\code{gamma.start}] The starting values for the second-stage causal model coefficients,
i.e., coefficients for the treatment variable and the model covariates.
This can either be a numerical value or a column vector with dimensions
equal to the number of second-stage coefficients.
The default value of NA will use the OLS estimate of second-stage coefficients as the starting value.
If this is a numerical value, that value will
serve as the starting value mean for all the second-stage gamma coefficients.

\item[\code{u.start}] The starting value for the precision hyperparameter of the inverse gamma distribution
(prior for the variance of the normal distribution of the first-stage residual term).
The default value of NA will use the inverse of the residual variance from the OLS estimate of the first-stage model.

\item[\code{e.start}] The starting value for the precision hyperparameter of the inverse gamma distribution
(prior for the scale parameter of Student's t distribution of the second-stage residual term).
The default value of NA will use the inverse of the residual variance from the OLS estimate
of the second-stage model.

\item[\code{df.start}] The starting value for the degrees of freedom of Student's t distribution.

\item[\code{n.chains}] Number of Markov chains. The default is 1.

\item[\code{n.burnin}] Length of burn in, i.e., number of iterations to discard at the beginning.
Default is n.iter/2, that is, discarding the first half of the simulations.

\item[\code{n.iter}] Number of total iterations per chain (including burnin). The default is 10000.

\item[\code{n.thin}] Thinning rate. Must be a positive integer. The default is 1.

\item[\code{DIC}] Logical; if TRUE (default), compute deviance, pD, and DIC. The rule pD=Dbar-Dhat is used.

\item[\code{codaPkg}] Logical; if FALSE (default), an object is returned; if TRUE,
file names of the output are returned.
\end{ldescription}
\end{Arguments}
%
\begin{Details}\relax
\begin{enumerate}

\item The formula takes the form \emph{response \textasciitilde{} terms|instrumental\_variables}.
\code{\LinkA{ts.nnormal}{ts.nnormal}} provides a detailed description of the formula rule.
\item DIC is computed as \emph{mean(deviance)+pD}.
\item Prior distributions used in ALMOND.
\begin{itemize}

\item Causal model coefficients at both stages: normal distributions.
\item The causal model residual at the first stage: normal distribution;
the causal model residual at the second stage: Student's t distribution.

\end{itemize}


\end{enumerate}

\end{Details}
%
\begin{Value}
If \emph{codaPkg=FALSE}(default), returns an object containing summary statistics of
the saved parameters, including
\begin{ldescription}
\item[\code{s1.intercept}] Estimate of the intercept from the first stage.
\item[\code{s1.slopeP}] Estimate of the pth slope from the first stage. 
\item[\code{s2.intercept}] Estimate of the intercept from the second stage.
\item[\code{s2.slopeP}] Estimate of the pth slope from the second stage (the first slope is always
the \strong{LATE}).
\item[\code{var.e.s1}] Estimate of the residual variance at the first stage.
\item[\code{var.e.s2}] Estimate of the residual variance at the second stage.
\item[\code{df.est}] Estimate of the degrees of freedom for the Student's t distribution.
\item[\code{DIC}] Deviance Information Criterion.
\end{ldescription}
If \emph{codaPkg=TRUE}, the returned value is the path for the output file
containing the Markov chain Monte Carlo output.
\end{Value}
%
\begin{Examples}
\begin{ExampleCode}

# Run the model
model1 <- ts.nrobust(outcome~treatment|instrument,data=subECLSK)

# Run the robust model with the self-defined advanced feature
my.robust.model<- function(){
  for (i in 1:N){
    mu[i] <- beta0 + beta1*z[i]
    x[i] ~ dnorm(mu[i], pre.u1)
    muY[i] <- gamma0 + gamma1*mu[i]
    y[i] ~ dt(muY[i], pre.u2, df)
  }

  beta0 ~ dnorm(0,1)
  beta1 ~ dnorm(1, 1)
  gamma0 ~ dnorm(0, 1)
  gamma1 ~ dnorm(.5, 1)

  pre.u1 ~ dgamma(.001, .001)
  pre.u2 ~ dgamma(.001, .001)

  df ~ dunif(0,50)

  s1.intercept <- beta0
  s1.slope1 <- beta1
  s2.intercept <- gamma0
  s2.slope1 <- gamma1
  df.est <- df
  var.e.s1 <- 1/pre.u1
  var.e.s2 <- 1/pre.u2
}

model2 <- ts.nrobust(outcome~treatment|instrument,data=subECLSK,
advanced=TRUE,adv.model=my.robust.model)

# Extract the model DIC
model1$DIC

# Extract the MCMC output
ts.nrobust(outcome~treatment|instrument,data=subECLSK,codaPkg=TRUE)

\end{ExampleCode}
\end{Examples}
\inputencoding{utf8}
\HeaderA{ts.nrobust.s}{Apply the Bayesian two-stage robust-selection causal model with instrumental variables.}{ts.nrobust.s}
%
\begin{Description}\relax
The \code{ts.nrobust.s} function applies the Bayesian two-stage robust-selection causal model
to the continuous treatment data. The model best suits the outcome data that contain outliers
and are nonignorably missing (i.e., MNAR) (e.g., dropout, attrition).
\end{Description}
%
\begin{Usage}
\begin{verbatim}
ts.nrobust.s(formula, data, m.ind, advanced = FALSE, adv.model, b0 = 1,
  B0 = 1e-06, g0 = 0, G0 = 1e-06, u0 = 0.001, U0 = 0.001,
  e0 = 0.001, E0 = 0.001, v0 = 0, V0 = 100, l0 = 0, L0 = 1e-06,
  beta.start = NULL, gamma.start = NULL, u.start = NULL,
  e.start = NULL, df.start = 5, lambda0.start = 1,
  lambda1.start = 1, n.chains = 1, n.burnin = floor(n.iter/2),
  n.iter = 50000, n.thin = 1, DIC, debug = FALSE, codaPkg = FALSE)
\end{verbatim}
\end{Usage}
%
\begin{Arguments}
\begin{ldescription}
\item[\code{formula}] An object of class formula: a symbolic description of the model to be fitted.
The details of the model specification are given under "Details".

\item[\code{data}] A dataframe with the variables to be used in the model.

\item[\code{advanced}] Logical; if FALSE (default), the model is specified using the formula argument,
if TRUE, self-defined models can be specified using the adv.model argument.

\item[\code{adv.model}] Specify the self-defined model. Used when advanced=TRUE.

\item[\code{b0}] The mean hyperparameter of the normal distribution (prior distribution)
for the first-stage causal model coefficients, i.e., coefficients for the instrumental variables.
This can either be a numerical value or a vector with dimensions equal to the number of coefficients
for the instrumental variables. If this takes a numerical value, then that values will
serve as the mean hyperparameter for all of the coefficients for the instrumental variables.
Default value of 0 is equivalent to a noninformative prior for the normal distributions.
Used when advanced=FALSE.

\item[\code{B0}] The precision hyperparameter of the normal distribution (prior distribution)
for the first stage causal model coefficients.
This can either be a numerical value or a vector with dimensions equal to the number of coefficients
for the instrumental variables. If this takes a numerical value, then that values will
serve as the precision hyperparameter for all of the coefficients for the instrumental variables.
Default value of 10E+6 is equivalent to a noninformative prior for the normal distributions.
Used when advanced=FALSE.

\item[\code{g0}] The mean hyperparameter of the normal distribution (prior distribution)
for the second-stage causal model coefficients,
i.e., coefficients for the treatment variable and other regression covariates).
This can either be a numerical value if there is only one treatment variable in the model,
or a if there is a treatment variable and multiple regression covariates,
with dimensions equal to the total number of coefficients for the treatment variable and covariates.
Default value of 0 is equivalent to a noninformative prior for the normal distributions.
Used when advanced=FALSE.

\item[\code{G0}] The precision hyperparameter of the normal distribution (prior distribution)
for the second-stage causal model coefficients.
This can either be a numerical value if there is only one treatment variable in the model,
or a vector if there is a treatment variable and multiple regression covariates,
with dimensions equal to the total number of coefficients for the treatment variable and covariates.
Default value of 10E+6 is equivalent to a noninformative prior for the normal distributions.
Used when advanced=FALSE.

\item[\code{u0}] The location hyperparameter of the inverse Gamma distribution (prior for the variance of the
normal distribution on the model residuals at the first stage).
Default of 0.001 is equivalent to the noninformative prior for the inverse Gamma distribution.

\item[\code{U0}] The shape hyperparameter of the inverse Gamma distribution (prior for the variance of the
normal distribution on the model residuals at the first stage).
Default of 0.001 is equivalent to the noninformative prior for the inverse Gamma distribution.

\item[\code{e0}] The location hyperparameter of the inverse Gamma distribution (prior for the scale parameter
of Student's t distribution on the model residuals at the second stage).
Default of 0.001 is equivalent to the noninformative prior for the inverse Gamma distribution.

\item[\code{E0}] The shape hyperparameter of the inverse Gamma distribution (prior for the scale parameter
of Student's t distribution on the model residuals at the second stage).
Default of 0.001 is equivalent to the noninformative prior for the inverse Gamma distribution.

\item[\code{v0}] The lower boundary hyperparameter of the uniform distribution (prior for the degrees of freedom
parameter of Student's t distribution).

\item[\code{V0}] The upper boundary hyperparameter of the uniform distribution (prior for the degrees of freedom
parameter of Student's t distribution).

\item[\code{l0}] The mean hyperparameter of the normal distribution (prior for the added-on selection model coefficients).
This can either be a numerical value or a vector with dimensions equal to the number of coefficients for the instrumental variables.
If this takes a numerical value, then that values will serve as the mean hyperparameter for all of the coefficients
for the instrumental variables. Default value of 0 is equivalent to a noninformative prior for the normal distributions.
Used when advanced=FALSE.

\item[\code{L0}] The precision hyperparameter of the normal distribution (prior for the added-on selection model coefficients).
This can either be a numerical value or a vector with dimensions equal to the number of coefficients for the instrumental variables.
If this takes a numerical value, then that values will serve as the precision hyperparameter for all of the coefficients
for the instrumental variables. Default value of 10E+6 is equivalent to a noninformative prior for the normal distributions.
Used when advanced=FALSE.

\item[\code{beta.start}] The starting values for the first-stage causal model coefficients,
i.e., coefficients for the instrumental variables.
This can either be a numerical value or a column vector with dimensions
equal to the number of first-stage coefficients.
The default value of NA will use the OLS estimate of first-stage coefficients as the starting value.
If this is a numerical value, that value will
serve as the starting value mean for all the first-stage beta coefficients.

\item[\code{gamma.start}] The starting values for the second-stage causal model coefficients,
i.e., coefficients for the treatment variable and the model covariates.
This can either be a numerical value or a column vector with dimensions
equal to the number of second-stage coefficients.
The default value of NA will use the OLS estimate of second-stage coefficients as the starting value.
If this is a numerical value, that value will
serve as the starting value mean for all the second-stage gamma coefficients.

\item[\code{u.start}] The starting value for the precision hyperparameter of the inverse gamma distribution
(prior for the variance of the normal distribution of the first-stage residual term).
The default value of NA will use the inverse of the residual variance from the OLS estimate of the first-stage model.

\item[\code{e.start}] The starting value for the precision hyperparameter of the inverse gamma distribution
(prior for the scale parameter of Student's t distribution of the second-stage residual term).
The default value of NA will use the inverse of the residual variance from the OLS estimate
of the second-stage model.

\item[\code{df.start}] The starting value for the degrees of freedom of Student's t distribution.

\item[\code{lambda0.start}] The starting value for the intercept of the coefficient of the added-on selection model.

\item[\code{lambda1.start}] The starting value for the slope of the coefficient of the added-on selection model.

\item[\code{n.chains}] The number of Markov chains. The default is 1.

\item[\code{n.burnin}] Length of burn in, i.e., number of iterations to discard at the beginning.
Default is n.iter/2, that is, discarding the first half of the simulations.

\item[\code{n.iter}] The number of total iterations per chain (including burnin). The default is 50000.

\item[\code{n.thin}] The thinning rate. Must be a positive integer. The default is 1.

\item[\code{DIC}] Logical; if TRUE (default), compute deviance, pD, and DIC. The rule pD=Dbar-Dhat is used.

\item[\code{codaPkg}] Logical; if FALSE (default), an object is returned; if TRUE,
file names of the output are returned.
\end{ldescription}
\end{Arguments}
%
\begin{Details}\relax
\begin{enumerate}

\item The formula takes the form \emph{response \textasciitilde{} terms|instrumental\_variables}.
\code{\LinkA{ts.nnormal}{ts.nnormal}} provides a detailed description of the formula rule.
\item DIC is computed as \emph{mean(deviance)+pD}.
\item Prior distributions used in ALMOND.
\begin{itemize}

\item Causal model coefficients at both stages: normal distributions.
\item The causal model residual at the first stage: normal distribution;
the causal model residual at the second stage: Student's t distribution.
\item Added-on selection model coefficients: normal distributions.

\end{itemize}


\end{enumerate}

\end{Details}
%
\begin{Value}
If \emph{codaPkg=FALSE}(default), returns an object containing summary statistics of
the saved parameters, including
\begin{ldescription}
\item[\code{s1.intercept}] Estimate of the intercept from the first stage.
\item[\code{s1.slopeP}] Estimate of the pth slope from the first stage. 
\item[\code{s2.intercept}] Estimate of the intercept from the second stage.
\item[\code{s2.slopeP}] Estimate of the pth slope from the second stage (the first slope is always
the \strong{LATE}).
\item[\code{select.intercept}] Estimate of the intercept from the added-on selection model.
\item[\code{select.slope}] Estimate of the slope from the added-on selection model.
\item[\code{var.e.s1}] Estimate of the residual variance at the first stage.
\item[\code{var.e.s2}] Estimate of the residual variance at the second stage.
\item[\code{df.est}] Estimate of the degrees of freedom for the Student's t distribution.
\item[\code{DIC}] Deviance Information Criterion.
\end{ldescription}
If \emph{codaPkg=TRUE}, the returned value is the path for the output file
containing the Markov chain Monte Carlo output.
\end{Value}
%
\begin{References}\relax
Gelman, A., Carlin, J.B., Stern, H.S., Rubin, D.B. (2003).
\emph{Bayesian data analysis}, 2nd edition. Chapman and Hall/CRC Press.

Spiegelhalter, D. J., Thomas, A., Best, N. G., Gilks, W., \& Lunn, D. (1996).
BUGS: Bayesian inference using Gibbs sampling.
\Rhref{http://www.mrc-bsu.cam.ac.uk/bugs}{}
\end{References}
%
\begin{Examples}
\begin{ExampleCode}

# Run the model
model1 <- ts.nrobust.s(outcome~treatment|instrument,data=simOutMNAR,m.ind=subECLSK$mis.ind,
n.iter=100000)

# Run the model with the self-defined advanced feature
my.robust.s.model<- function(){
  for (i in 1:N){
    mu[i] <- beta0 + beta1*z[i]
    x[i] ~ dnorm(mu[i], pre.u1)
    muY[i] <- gamma0 + gamma1*mu[i]
    y[i] ~ dt(muY[i], pre.u2, df)

    m[i] ~ dbern(q[i])
    q[i] <- phi(lambda0 + lambda1*y[i])
  }

  beta0 ~ dnorm(0,1)
  beta1 ~ dnorm(1, 1)
  gamma0 ~ dnorm(0, 1)
  gamma1 ~ dnorm(.5, 1)
  lambda0 ~ dnorm(0, 1.0E-6)
  lambda1 ~ dnorm(0, 1.0E-6)

  pre.u1 ~ dgamma(.001, .001)
  pre.u2 ~ dgamma(.001, .001)

  df ~ dunif(0,50)

  s1.intercept <- beta0
  s1.slope1 <- beta1
  s2.intercept <- gamma0
  s2.slope1 <- gamma1
  select.intercept <- lambda0
  select.slope <- lambda1
  df.est <- df
  var.e.s1 <- 1/pre.u1
  var.e.s2 <- 1/pre.u2
}

model2 <- ts.nrobust.s(routcome~treatment|instrument,data=simOutMNAR,
m.ind=subECLSK$mis.ind, advanced=TRUE,adv.model=my.robust.s.model,n.iter=100000)

# Extract the model DIC
model1$DIC

# Extract the MCMC output
ts.nrobust.s(outcome~treatment|instrument,data=simOutMNAR,m.ind=subECLSK$mis.ind,
codaPkg=TRUE)


\end{ExampleCode}
\end{Examples}
\printindex{}
\end{document}
